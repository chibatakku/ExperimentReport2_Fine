\section{考察}
本実験をとおして脆性材料はヤング率が高く,アルミニウム並みの剛性があることが分かった.これは荷重に対してほとんど変形しないことを示している.また,破断に至るまでの変形量が小さく,靭性に乏しいことも改めて認識できた.

傷なしと傷なしでは,曲げ弾性係数,破断応力ともに傷ありのほうが低い値を示した.これは,傷があることで応力集中が生じ,亀裂が進展しやすくなるためだと考えられる.

破壊データがばらついた原因として,ガラス表面の傷の状態がガラスによって異なることがあげられる.手作業で傷をつけたガラス片はもちろん,傷をつけていないガラス片にも実際には,目に見えない大きさの傷が数多く存在する.ガラスのような脆性材料では,このような微小な傷による応力集中に敏感であり,破壊強度に影響を及ぼしたと考えられる.

セラミックスの応用例として,セラミック工具があげられる.セラミックは高い硬度と優れた耐熱特性,化学的安定性を有しているのが特徴であり,高速切削に適している.セラミックの主な成分としては,アルミナ,窒化ケイ素,サイアロンなどがある.
セラミック最大の欠点は,強度が低く信頼性が小さいことである.大きな力,特に衝撃がかかる場合には容易に破損するため,軽切削などに用途が限られる.そこで,基材に強度の高い工具鋼や超硬合金を用い,その表面をセラミック化することで,耐摩耗性が高くかつ欠けにくい工具にするセラミック・コーティングが行われている.