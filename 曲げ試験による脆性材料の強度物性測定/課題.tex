\section{課題}

対称性から $0 \leq x \leq l_0 + \frac{1}{2}l_1$ の範囲で考えればよい。

$0 \leq x \leq l_0$ でたわみ曲線の微分方程式は

\begin{equation}
    \frac{d^2y}{dx^2} = -\frac{W}{EI}x
    \label{eq:たわみ式1}
\end{equation}

\(x\) について積分すると
\begin{equation}
    \frac{dy}{dx} = -\frac{W}{2EI}x^2 + C_1
    \label{eq:傾き1}
\end{equation}
\begin{equation}
    y = -\frac{W}{6EI}x^3 + C_1x + C_2
    \label{eq:変位1}
\end{equation}
$x = 0$ で $y = 0$ より $C_2 = 0$ である。

$l_0 \leq x \leq l_0 + \frac{1}{2}l_1$ でたわみ曲線の微分方程式は
\begin{equation}
    \frac{d^2y}{dx^2} = -\frac{W}{EI}l_0
    \label{eq:たわみ式2}
\end{equation}

\(x\) ついて積分すると
\begin{equation}
    \frac{dy}{dx} = -\frac{Wl_0}{EI}x + C_3
    \label{eq:傾き2}
\end{equation}
\begin{equation}
    y = -\frac{Wl_0}{2EI}x^2 + C_3x + C_4
    \label{eq:変位2}
\end{equation}

$x = l_0 + \frac{1}{2}l_1$ で $\frac{dy}{dx} = 0$ より
\begin{equation}
    C_3 = \frac{Wl_0}{EI}(l_0 + \frac{1}{2}l_1)
    \label{eq:C3}
\end{equation}

$x = l_0$ で,式 (\ref{eq:傾き1}) と式 (\ref{eq:傾き2}) が連続であるから

\begin{align}
    -\frac{W}{2EI}l_0^2 + C_1 &= -\frac{Wl_0}{EI}l_0 + \frac{Wl_0}{EI}(l_0 + \frac{1}{2}l_1)\\
    C_1 &= \frac{Wl_0}{2EI}(l_0 + l_1)
    \label{eq:連続条件1}
\end{align}

以上から得られた$C_1$と$C_2$を式(\ref{eq:変位1})に代入すると
\begin{equation}
    y = -\frac{W}{6EI}x^3 + \frac{Wl_0}{2EI}(l_0 + l_1)x
    \label{eq:変位1_2}
\end{equation}
$x = l_0$のとき
\begin{equation}
    y = \frac{W}{6EI}(2l_0^3 + 3l_1l_0^2)
    \label{eq:変位1_3}
\end{equation}