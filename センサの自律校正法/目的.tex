\section{実験の背景と目的}
精密測定において,より良い正しいものを基準にした比較測定によって精度を保証することが基本となる.しかし,入手しうる基準を超えた測定精度を達成するためには,基準に頼らずに自律的に精度を向上させる技術が重要となる.実験対象とした静電容量型変位センサは,10nmより高い分解能を有する比較的安定した使いやすい変位センサであるが,分解能の限界近くでは線形誤差が無視できない程大きくなる.
本実験では,変位センサの線形誤差を自律的に校正する手法について学ぶとともに,各種センサ,測定器を正しく使用するための知識の習得や比較測定と基準の関係についての理解を深めること,変位センサに含まれる線形誤差を調べることなどを目的とする.