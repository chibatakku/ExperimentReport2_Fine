\section{実験方法}

\subsection{ノイズレベル及びドリフトの測定}
ノイズレベルはシステムの分解能の限界と深く関係があり,実験中のドリフト量は測定結果の不確かさに影響する.そこで自律校正実験に先立ち,ノイズレベルとドリフト量を測定した.また,温度の影響を見るためにセンサ近傍の温度変化も同時に記録した
.

\subsection{自律校正実験}
100$\mathrm{\mu m}$の測定範囲を持つ静電容量型変位センサの校正曲線を,2$\mathrm{\mu m}$ごと,50点で表現した.実験装置のてこの倍率は$n = 5$である.まず,センサAを基準にセンサBの校正を行い,つぎにセンサBを基準にセンサAの校正を行った.

自律校正実験は以下の手順で行った.
\begin{enumerate}
    \item センサAが基準側にセンサBが被校正側にセットされていることを確認した.
    \item 基準側,被校正側の変位センサの読みが-9.5になるようにステージ1,2を動かした.
    \item その後,マイクロメータヘッド3を回し,はりに曲げを与えてゆく.基準側変位センサ出力電圧を-8,-6,・・・,+8,+9.5Vと2Vずつ変化させ,その度に「測定ボタン」をクリックし,被校正側と基準センサの出力を同時に取り込んだ.ただし,最後ははりとセンサの衝突を防ぐため,+9.5Vとした.
    \item 基準側のセンサが9.5Vに達したら基準側ステージ1を用いてはりから離す向きに動かし,基準側センサの出力電圧を最小値(-9.5V)まで戻した.調整が終わったら「測定ボタン」を押した.
    \item 被校正側の全範囲にわたる読みを取るまで,(2)-(4)を繰り返す.なお,変位センサAの読みから変位を求めるには分かっている校正曲線を使った.
    \item 次にセンサの位置を入れ替えた.センサBを基準側に,センサAを被校正側に配置した.
    \item センサを入れ替えた直後は出力が安定しないため,安定性を確認するためドリフト測定を行った.
    \item (2)-(5)を繰り返した.
\end{enumerate}