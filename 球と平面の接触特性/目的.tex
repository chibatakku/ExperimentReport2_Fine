\section{目的}
二つの固体平面の接触において,固体の表面に全く粗さが存在しないとすれば,二つ
の固体面は全面的に直接接触する.このような接触状態を面接触と呼ぶ.しかし,実際
の工学的表面は表面粗さを有するため,接触は突起頂点同士で順次起こり,接触した突
起部が荷重によって弾性変形あるいは塑性変形し,接触変形部分が荷重を支えると考え
られる.このような接触状態を分散接触と呼ぶ.分散接触状態において実際に接触して
いる面積(真実接触面積)は,見かけの接触面積に比べ極めて小さく,場合によっては,見かけの接触面積の数万分の 1 程度になることもある.したがって,わずかな面積に荷重が集中し,極めて高い接触圧力が発生することになり,これらが表面損傷の原因となる場合もある.このような接触問題は,転がり軸受や歯車のような,外力接触下において転がり/すべり運動をする機械要素の設計において重要である.また,接触問題は,摩擦や潤滑のみならず,熱伝導,電気接点,機械加工,塑性加工などの問題を扱う場合にまず解決しなければならない重要課題である.本実験では,接触問題の基本であるヘルツの弾性接触理論を理解し,球と平面の静的接触形態と摩擦係数との関係を実験的に明らかにすることにより,接触問題に対する理解を深めることを目的とする.