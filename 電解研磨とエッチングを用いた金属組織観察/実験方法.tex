\section{実験方法}
本実験で用いた材料は,(1)高Cr鋼,(2)SUS304ステンレス鋼,(3)銅である.鋼Crフェライト鋼は異なる疲労負荷条件で破断させた試料である.また,オーステナイト系ステンレス鋼は熱処理により,粒界にCr欠乏層が形成された鋭敏化材もの含む.配布された試料に基づき,以下の実験手順で観察を行った.

\subsection{高Cr鋼}
\begin{enumerate}
    \item 機械的研磨\\
          研磨機を用いてバフ研磨を行い試料表面を平坦にした.\\
          荷重:5N\\
          研磨液:ダイアモンド粒径3$\mu$m(1回目)→1$\mu$m(2回目)\\
          研磨時間:300秒\\
          回転数:バフ120rpm\\
                 ヘッド60rpm
    \item 化学的エッチング\\
          ピロ亜硫酸ナトリウム(Na$_2$S$_2$O$_5$)13g,チオ硫酸ナトリウム(Na$_2$S$_2$O$_3$)10g,クエン酸(C$_6$H$_8$O$_7$)1.5g,エタノール5ml,水50mlからなる腐食液を使用した.\\
          エッチング時間:300秒
      \item 光学顕微鏡で表面組織を観察した.
\end{enumerate}

\subsection{SUS304ステンレス鋼}
試験片は(i)MA,(ii)650$^\circ$C2時間,(iii)650$^\circ$C2時間 + 850$^\circ$C2時間の熱処理を施したもので行った.
\begin{enumerate}
      \item 機械的研磨\\
            研磨機を用いてバフ研磨を行い試料表面を平坦にした.\\
            荷重:5N\\
            研磨液:ダイアモンド粒径3$\mu$m(1回目)→1$\mu$m(2回目)\\
            研磨時間:300秒\\
            回転数:バフ120rpm\\
                   ヘッド60rpm
      \item 電気化学エッチング\\
            試験片のエッチング面を陽極として10\%シュウ酸溶液中に入れ,エッチング面積1cm$^2$あたりの電流を1Aに調整して90秒エッチングを行った.1Aにしたときの電圧は5.0Vであった.
      \item 光学顕微鏡で表面組織を観察した.
\end{enumerate}

\subsection{銅}
\begin{enumerate}
    \item ダイアモンドペーストによって機械研磨を行った.
    \item 電解研磨を以下の条件で行った.\\
          電解研磨液:リン酸銅浴(リン酸濃度2/3)\\
          電圧:3.6V\\
          電流:600mA\\
          印加時間:180秒
    \item 光学顕微鏡で表面組織を観察した.
\end{enumerate}
\clearpage