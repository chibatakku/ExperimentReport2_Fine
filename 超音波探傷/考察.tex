\section{考察}

\subsection{超音波き裂探傷法に内在する問題点}
超音波き裂探傷法の欠点として,振動によるノイズの影響を受けやすいことがあげられる.これを軽減するためには,周囲でものを動かさない,しゃべらない,温度を一定にするといった環境の整備が重要である.また,手作業では手の震えや体温による温度変化,探触子を押さえつけてしまうことによる試験体の形状変化などによってノイズが生じるため,高精度な測定ではxyz3軸のステージを備えた装置を用いることが望ましいと思われる.さらに,自動化によって測定するような場合には,半導体露光装置で用いられているような,カウンターウェイトによる振動キャンセルなどの制御を行うことでより高速で高精度な測定が行えるのではと考える.

\subsection{垂直探傷法以外の測定法}
パルス反射法はには,垂直探傷法のほかに斜格探傷法,表面波探傷法,板波探傷などがある.

斜角探傷法は被検体の探傷面に45度,60度,70度などの角度をつけて超音波を入射する方法である.溶接部などの垂直探傷法が困難な場所で,内部検査や表面割れ検査に使用される.

表面波探傷法は被検体の表面のみに超音波を伝搬し,表面傷を高感度で検出する方法であり,以下のような特徴を持つ.
\begin{itemize}
    \item 超音波のエネルギーが表面に集中するため,距離減衰が小さい.
    \item 表面に水滴が付着しても検出してしまうため,誤差判定対策が必要.
    \item 表面波のもぐりこみ深さは1~1/2波長で,2MHz以下の比較的低い周波数を使用することが多い.
    \item 横波の臨界角近傍で表面波が発生する.
\end{itemize}

板波とは比較的薄い個体平面層内を平面に沿って伝播する弾性体のことである.板波探傷法は板波を利用したもので,単一の探触子によって全面検査が可能である.この理由により,圧延工程で導入され,品質管理の一方法として利用されている.この探傷法には板厚,欠陥位置,周波数,板波モードなどによって欠陥検出感度が大きく変化する特徴がある.
\subsection{超音波の応用例}
超音波の応用例として以下のようなものがある.
\begin{itemize}
    \item 超音波洗浄:超音波の物理的作用(キャビテーション,振動加速度,直進流)と洗浄液による化学的作用及び超音波による化学反応促進作用によって,洗浄効果を得ている.
    \item 超音波加工:超音波加工には,砥粒加工,切削加工,接合,溶着などがある.特に切削加工では,工具にかかる力が少なくなることによる加工精度の向上や,発熱による加工対象物の変質などを防ぐなどといったメリットがある.
    \item 魚群探知機:魚群探査は探傷法と似たような原理で,海底の反射波と魚群の反射波から,魚群の位置や大きさを知ることができる.
\end{itemize}

超音波の利用法として,fff式の3Dプリンターが考えられる.ノズルに超音波振動させることにより,ノズル詰まりや糸引きの防止などが期待できる.また,ビルドプレート側を振動させることで,簡単に造形物の取り出しをできるようになり,ビルドプレートの寿命を延ばすことができる可能性もある.さらに,超音波振動による汚れの除去によって,造形物の積層強度や品質の向上といったことも期待できる.


\subsection{欠陥の大きさの推定}
まずSTB-G・V15-2は$\phi$2mmの円形平面傷を持つ標準試験片であり,探触子の直径は30mmであるから,試験片のGは以下のように求められる.
\begin{equation}
    \label{eq:試験片G}
    G = \frac{2}{20} \approx 0.0667
\end{equation}
また,試験片の音速を5950m/sとすると$z_F = 150$mmより$n_F$は以下のようになる.
\begin{equation}
    \label{eq:試験片nF}
    n_F = \frac{4 \cdot 150 \cdot 5950 \cdot 10^{-3}}{30^2 \cdot 2} \approx 1.98
\end{equation}
よって,AVG線図から$F/B_0\mathrm{[dB]} = -40$となる.鍛鋼品は6dB高いため$F/B_0\mathrm{[dB]} = -34$である.鍛鋼品において$z_F = 200$より,音速を5950m/sとして$n_F$は以下のように求められる.
\begin{equation}
    \label{eq:鍛鋼品nF}
    n_F = \frac{4 \cdot 200 \cdot 5950 \cdot 10^{-3}}{30^2 \cdot 2} \approx 2.64
\end{equation}
以上からAVG線図よりGを過大評価して求めると,$G = 0.15$.したがって,鍛鋼品における欠陥の大きさは以下のように推定できる.
\begin{equation}
    \label{eq:鍛鋼品d}
    d = 0.15 \cdot 30 = 4.5\mathrm{[mm]}
\end{equation}
