\section{考察}
表\ref{tbl:応力値と信頼限界}より,塑性変形前の残留応力はA点,B点のどちらも圧縮で,A点の方が大きい値を示した.塑性変形後の残留応力はA点ではやや値が下がったものの圧縮のままであったが,B点では引張応力に変化した.これは曲げ加工の外側であるA点では加工時に引張応力が加わるため,材料内部では反発として圧縮応力が残留し,一方内側であるB点では加工時に圧縮応力が加わるため,反発として引張応力が残留した結果であると考えられる.

