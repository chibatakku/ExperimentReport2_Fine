\section{残留応力測定法について}

残留応力の測定法を以下に示す.

\subsection{X線回折}
金属材料に応力を加えると結晶格子がひずみ,本来の格子間距離より伸縮する.この時X線回折がBragg条件が式(\ref{eq:braggの式})で生じるため,これから格子間距離を求め,残留応力を算出することができる.X線応力測定は非破壊,非接触で表面微小領域の測定や応力分布の可視化,3軸応力解析による主応力の評価などが可能である.
一方,試料表面層の応力しか測定できない欠点があるほか,

\subsection{ひずみゲージ法}

\subsection{中心除去法}
