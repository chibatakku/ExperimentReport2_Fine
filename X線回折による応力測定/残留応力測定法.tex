\section{残留応力測定法について}

残留応力の測定法を以下に示す.

\subsection{X線回折}
金属材料に応力を加えると結晶格子がひずみ,本来の格子間距離より伸縮する.この時X線回折がBragg条件が式(\ref{eq:braggの式})で生じるため,これから格子間距離を求め,残留応力を算出することができる.X線応力測定は非破壊,非接触で表面微小領域の測定や応力分布の可視化,3軸応力解析による主応力の評価などが可能である.
一方,試料表面層の応力しか測定できないことが欠点としてあげられる.

\subsection{ひずみゲージ法}
まず,測定したい個所数方向にひずみゲージを貼り,その部分のひずみを測定.その後,その部分を含んだ小片を切り出し,再度ひずみ測定を行うことで前者との差を求め,応力を測定する方法である.ひずみゲージ法の利点はひずみを直接測定できることや,測定装置が比較的簡単であることなどがあげられる.一方欠点は破壊検査であることであり,小片を切り出す際の塑性変形がひずみゲージに影響を及ぼさないよう考慮する必要がある.
\subsection{穿孔法}
残留応力を有する物体に穴をあけると,その位置で応力が解放される.穴表面でのせん断応力や垂直応力の除去は,すぐ近くの周囲領域の応力を変化させ,それに対応して試験対象物の表面の局所ひずみを変化させる.このひずみをひずみゲージで測定することで残留応力を求めることができる.穿孔法の大きな利点は材料内部の応力分布を計測できることである.欠点は破壊検査であることで,ドリル加工により穴をあけると加工硬化が生じるため,その影響を考慮する必要がある.
