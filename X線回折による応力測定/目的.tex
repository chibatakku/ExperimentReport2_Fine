\section{目的}
機器・構造物の代表的な損傷である疲労き裂や応力腐食割れなどには,部材中に存在する残留応力が深く関わっている.残留応力とは,外力が作用しないとき部材内部で釣り合いを保って存在する応力である.機器・構造物には残留応力と外力による応力が重畳して負荷されるので,一般に圧縮残留応力が存在する場合には引張残留応力が存在する場合に比べて,疲労強度が向上する.また残留応力が引張の場合には応力腐食割れが生じるが,圧縮の場合には生じない.代表的な残留応力計測法の一つにX線回折を用いた手法があり,機械材料に負荷された応力も計測できる.本実験では,X線回折のなかで一般的な手法である $sin^2\psi$法により残留応力ならびに負荷応力を計測する.