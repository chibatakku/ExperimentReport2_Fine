\documentclass[autodetect-engine, dvi=dvipdfmx, ja=standard, a4paper, 12pt]{bxjsarticle}

\usepackage[dvipdfmx]{graphicx}

\usepackage{amsmath}
\usepackage{amssymb,latexsym,mathtools}
\usepackage{tabularx}
\usepackage{bm}
\usepackage{url}

%サブキャプションを使うとき
\usepackage[hang,small,bf]{caption}
\usepackage[subrefformat=parens]{subcaption}


\usepackage[version=4]{mhchem} %千葉匠追加
\usepackage{multirow}
\usepackage{bm}
\usepackage{physics}
\usepackage{amsmath}
\usepackage{textcomp} % \degree記号のため

% 表の番号のフォーマットを "Table 1" という形式にする
\renewcommand{\tablename}{Table}


% 図の番号のフォーマットを"Fig.1"という形式にする
\renewcommand{\figurename}{Fig.}

% タイトルの読み込み
%
% 卒業論文・発表要旨
% 
%

% 卒業年度(元号で)
\def\年度{令和6年度}

% 論文の種類:本科→卒業論文 専攻科→修了論文
% 以下のどちらかのコメントを外して下さい.
\def\論文種類{機械知能・航空実験II} % 本科の場合
%\def\論文種類{修了論文} % 専攻科

\def\班{A班}

% 発表会日時(要旨用)
%\def\発表会日時{令和5年2月28日}

% 研究題目
\def\研究題目{ファイン2 \\超音波探傷}
\def\研究題目英語{ }

% 副題
%\def\研究題目副{~もげもげに関する考察~}  % 前後に 飾り"~" をつける
\def\研究題目副{ } % 副題がない場合は全角スペース
%\def\研究題目副英語{-- Considering about moge-moge --} % 前後に飾り "--" をつける
\def\研究題目副英語{ }% 副題(英語)がない場合

% 学生氏名,所属等
\def\学籍番号{C2TB1505}
\def\所属{機械知能・航空工学科 \\ファインメカニクスコース 高・松隈研究室} % 機械知能・航空工学科
\def\学生氏名{千葉 匠}
\def\学生氏名英語{TAKUMI CHIBA}  %名字は全て大文字に
\def\共同実験者{川口朋也,蔦森公亨,吉村悠太}
%指導教員氏名
%\def\指導教員氏名{ }
%\def\指導教員氏名英語{ }
% タイトル用日付
\def\実験日{2024年11月13日}
\def\レポート提出日付{2024年11月20日} %西暦+月として下さい
\def\連絡先{chiba.takumi.s4@dc.tohoku.ac.jp}




\begin{document}

% タイトル生成
%
% 卒業論文 表紙
%
% 一関高専 機械・知能系 藤原康宣
% 『卒業論文.tex』を処理すると読み込まれて論文の表紙を生成します
% ※このファイルは編集する必要はありません.
%


\begin{titlepage}
    \begin{center}
    \vspace*{20truept}
    {\LARGE \textgt{\年度 \論文種類 \班}}\\
    \vspace*{60truept}
    
    % 題目出力
    {\Huge \textgt{\研究題目}}\\ %研究題目
    \vspace{10truept}
    {\LARGE  \textgt{ \研究題目副} }\\ % 副題
    \vspace{30truept}
    
    %題目(英語)出力
    {\Large \textbf{\研究題目英語}}\\
    {\large \textbf{\研究題目副英語 }}\\
    
    % 学生氏名出力
    
    {\LARGE \textgt{東北大学 \所属}}\\ % 所属
    \vspace{1zw}
    {\Large \textgt{学籍番号 \学籍番号}}\\ % 学籍番号
    \vspace{1zw}
    {\Huge \textgt{\学生氏名}}\\ % 著者
    \vspace{1zw}
    {\Large \textgt{共同実験者 \共同実験者}}\\
    \vspace{3zw}
    
    % 指導教員氏名出力
    %{\LARGE \textgt{指導教員 \指導教員氏名}}\\ % 著者
    %\vspace{3zw}
    
    % 提出日出力
    {\Large \textgt{実験日 \実験日}}\\
    \vspace{1zw}
    % {\Large \textgt{作成日 \作成日}}\\
    % \vspace{1zw}
    {\Large \textgt{提出日 \レポート提出日付}}\\ % 提出日
    \vspace{3zw}
    {\Large \textgt{連絡先 \連絡先}}\\
    \end{center}
    \end{titlepage}


% 目次生成
\tableofcontents 
\clearpage
% 章ごとに別ファイルにして,下さい.
% 各章のファイルは,\input{ファイル名} で読み込んで下さい
%\input{章別サンプル}
\section{目的}
本実験では,脆性材料のとしてガラスを選択して曲げ試験を行い,強度分布を考慮した強度評価法を習得するとともに,脆性材料の機械的性質を理解することを目的とする.
\section{自律校正の原理}
図\ref{fig:校正系}に自律校正を行う装置校正を示す.同じセンサを2つ用意し,変位を与えるレバーを用意する.この時支点からzの位置に校正するセンサBを配置し,nzの位置に基準とするセンサAを配置する.すると基準側の変位は被校正側のn倍となる.
\begin{figure}[htbp]
    \centering %中央揃え
    \includegraphics[width=100truemm,clip]{fig/fig_校正系.png}
    \caption{Autonomous calibration device configuration.}
    \label{fig:校正系}
\end{figure}

図\ref{fig:step1}に校正のstep1(n = 2)を示す.まず基準側のセンサAで平均感度から変位$X_{Mea}$を求める.True curveを仮定すると誤差$e_{A0}$は次式で求められる.
\begin{equation}
    e_{A0} = X_{True} - X_{Mea}
\end{equation}
次に$X_{Mea}$の値を用いてセンサBの校正を行う.被校正側の変位は基準側の1/nであるから誤差$e_{B1}$は次式で求められる.ここで変位が基準側の1/nになることから,レバーを動かすことで基準側ではn回の測定を行うことになる.
\begin{equation}
    e_{B1} = \frac{X_{True}}{n} - \frac{X_{Mea}}{n} = \frac{e_{A0}}{n}
    \label{eq:誤差B1}
\end{equation}

\begin{figure}[htbp]
    \centering %中央揃え
    \includegraphics[width=120truemm,clip]{fig/fig_step1.png}
    \caption{Calibration of sensor B by sensor A(step1).}
    \label{fig:step1}
\end{figure}
図\ref{fig:step2}に校正のstep2(n = 3)を示す.センサAとBの位置を入れ替え,センサBを基準側とする.センサBの校正曲線から誤差は式(\ref{eq:誤差B1})で求められる.次にセンサBの$X_{Mea}$を用いて,センサAの校正を行う.誤差$e_{A2}$は次式で求められる.
\begin{equation}
    e_{A2} = \frac{X_{True}}{n} - \frac{X_{Mea}}{n} = \frac{e_{B1}}{n} = \frac{e_{A0}}{n^2}
    \label{eq:誤差A2}
\end{equation}
\begin{figure}[htbp]
    \centering %中央揃え
    \includegraphics[width=120truemm,clip]{fig/fig_step2.png}
    \caption{Calibration of sensor A by sensor B(step2).}
    \label{fig:step2}
\end{figure}

以上の手順をk回繰り返すと,誤差$e_{Ak-1}$,$e_{Bk}$は次式で求められる.
\begin{equation}
    e_{Ak-1} = \frac{e_{A0}}{n^{k-1}}
    \label{eq:誤差Ak-1}
\end{equation}
\begin{equation}
    e_{Bk} = \frac{e_{A0}}{n^{k}}
    \label{eq:誤差Bk}
\end{equation}
このようにkが大きくするにつれて誤差が小さくなっていることが分かる.これが自律校正の原理である.
\section{実験装置}
図\ref{fig:実験装置}に変位センサとレバー系からなる実験装置の概略を示す.校正しようとする2本の変位センサはそれぞれ一軸ステージ1,2に載せられる.ステージを動かすことによって,変位センサのレバーに対する位置が調整できる.またレバーはマイクロメータヘッド3によって駆動される.なお,それぞれのステージやマイクロメータヘッドは手動で操作した.
\begin{figure}[htbp]
    \centering %中央揃え
    \includegraphics[width=100truemm,clip]{fig/fig_実験装置.png}
    \caption{Schematic diagram of experimental apparatus.}
    \label{fig:実験装置}
\end{figure}
\section{実験方法}
本実験では以下の3つの実験を行った.

\renewcommand{\theenumi}{\Roman{enumi}}
\begin{enumerate}
    \item 平面き裂:6デシベルドロップ法
    \item 欠陥評価:AVG線図
    \item 最先端超音波探傷の理解
\end{enumerate}

実験I,IIで用いた実験装置を図\ref{fig:実験装置}に示す.パルサーは探触子を振動させ超音波を発信する装置であり,レシーバーは反射した超音波を探触子で電気信号に表示する装置である.探触子は周波数$f = 5\mathrm{MHz}$,直径$D = 6.3\mathrm{mm}$のものを使用した.
\begin{figure}[htbp]
    \centering %中央揃え
    \includegraphics[width=100truemm,clip]{fig/実験装置.png}
    \caption{Experimental apparatus in Experiments I and II.}
    \label{fig:実験装置}
\end{figure}

\renewcommand{\theenumi}{\arabic{enumi}}
\subsection{実験I}
図\ref{fig:実験I}に計測した試験体の形状を示す.実験手順を以下に示す.
\begin{enumerate}
    \item 試験体表面にグリスを塗布した.
    \item 探触子を試験体表面に当てて,オシロスコープのエコーを確認した.
    \item 第1回エコーが送信パルスに対して,大きく変わらない場所で探触子を静止し,オシロスコープの波形を記録した.
    \item オシロスコープの波形から図\ref{fig:試験体厚さ}に示すような底面往復時間$t_B(\mathrm{\mu s})$と底面エコー高さ$B$(V)を測定した.
    \item 探触子を動かし,第1回エコー高さが半減した場所でオシロスコープの波形を記録し,探触子の位置を欠陥先端位置$L$(mm)として測定した.
    \item オシロスコープの波形から図\ref{fig:欠陥深さ}に示すような欠陥往復時間$t_F(\mathrm{\mu s})$と欠陥エコー高さ$F$(V)を測定した.
    \item 式(\ref{eq:試験片厚さ}),式(\ref{eq:欠陥深さ})から試験片厚さ$Z_B$と欠陥深さ$z_F$を計算した.ここで$C$は試験片の縦波音速である.
    \item 以上の手順を4回繰り返した.
\end{enumerate}
\begin{figure}[htbp]
    \centering %中央揃え
    \includegraphics[width=100truemm,clip]{fig/実験I.png}
    \caption{Shape of the specimen in Experiment I.}
    \label{fig:実験I}
\end{figure}
\begin{figure}[htbp]
    \centering %中央揃え
    \includegraphics[width=100truemm,clip]{fig/試験体厚さ.png}
    \caption{Measurement of specimen thickness.}
    \label{fig:試験体厚さ}
\end{figure}
\begin{figure}[htbp]
    \centering %中央揃え
    \includegraphics[width=100truemm,clip]{fig/欠陥深さ.png}
    \caption{Measurement of defect depth.}
    \label{fig:欠陥深さ}
\end{figure}
\begin{equation}
    \label{eq:試験片厚さ}
    z_B = \frac{1}{2}Ct_B
\end{equation}
\begin{equation}
    \label{eq:欠陥深さ}
    z_F = \frac{1}{2}Ct_F
\end{equation}

\subsection{実験II}
実験IIでは試験片Al1,Al2,ステンレスの3種類に対して測定を行った.実験手順を以下に示す.

\begin{enumerate}
    \item 実験Iと同様にしてそれぞれの試験片で,底面エコー高さ$B(V)$と底面往復時間を測定した.Al1を2回,Al2とステンレスを1回ずつ測定し,それぞれの平均を求めた.
    \item 底面往復時間$t_B$より試験片の厚さ$Z_B$を求めた.
    \item 求めた$z_B$より底面までの音波路程$n_B$を式(\ref{eq:音波路程})から計算した.ここで波長$\lambda$は$\lambda = C/f$である.
    \item 求めた$n_B$と図\ref{fig:AVG線図}から読み値$= 20log_{10} \frac{B}{B_0}$より,基準化したエコー高さ$B_0$を求めた.
    \item 次にAl1のき裂3点において,欠陥の位置$x, y$,欠陥エコー高さ$F$(V),欠陥往復時間$t_F$を測定した.3点目の欠陥のみ2回測定した.
    \item Al2及びステンレスは4つの欠陥においてAl1と同様に測定を行った.
    \item 測定結果から欠陥深さ$z_F$,欠陥エコー高さ$F$,欠陥音波路程$n_F$を計算した.さらに$F/B_0$を計算し,図\ref{fig:AVG線図}からGを過大評価により求め,式(\ref{eq:欠陥直径})より欠陥直径$d$を推定した.
\end{enumerate}
\begin{equation}
    \label{eq:音波路程}
    n_B = \frac{4z_B\lambda}{D^2}
\end{equation}
\begin{equation}
    \label{eq:欠陥直径}
    d = G \cdot D
\end{equation}

\subsection{実験III}
実験IIIでは,図\ref{fig:超音波映像装置}に示すような超音波映像装置による最先端の超音波技術により,suicaの回路構造が表示される様子を観察した.

\begin{figure}[htbp]
    \centering %中央揃え
    \includegraphics[width=100truemm,clip]{fig/超音波映像装置.png}
    \caption{Ultrasonic imaging device.}
    \label{fig:超音波映像装置}
\end{figure}
\section{実験結果}
図にバフ研磨を行った高Cr鋼の表面の撮影結果を示す.

\begin{figure}[htbp]
    \begin{minipage}[htbp]{0.45\linewidth}
      \centering
      \includegraphics[keepaspectratio, scale=0.07]{fig/241218_9Cr_3um.jpg}
      \subcaption{Diamond paste 3$\mu$m}
      \label{fig:Cr3um}
    \end{minipage}
    \begin{minipage}[htbp]{0.45\linewidth}
      \centering
      \includegraphics[keepaspectratio, scale=0.07]{fig/241218_9Cr_1um.jpg}
      \subcaption{Diamond paste 1$\mu$m}
      \label{fig:Cr1um}
    \end{minipage}
    \centering
    \caption{Surface of high Cr steel after buffing.}
    \label{fig:CrBuff}
\end{figure}
    

\section{考察}
本実験をとおして脆性材料はヤング率が高く,アルミニウム並みの剛性があることが分かった.これは荷重に対してほとんど変形しないことを示している.また,破断に至るまでの変形量が小さく,靭性に乏しいことも改めて認識できた.

傷なしと傷なしでは,曲げ弾性係数,破断応力ともに傷ありのほうが低い値を示した.これは,傷があることで応力集中が生じ,亀裂が進展しやすくなるためだと考えられる.

破壊データがばらついた原因として,ガラス表面の傷の状態がガラスによって異なることがあげられる.手作業で傷をつけたガラス片はもちろん,傷をつけていないガラス片にも実際には,目に見えない大きさの傷が数多く存在する.ガラスのような脆性材料では,このような微小な傷による応力集中に敏感であり,破壊強度に影響を及ぼしたと考えられる.

セラミックスの応用例として,セラミック工具があげられる.セラミックは高い硬度と優れた耐熱特性,化学的安定性を有しているのが特徴であり,高速切削に適している.セラミックの主な成分としては,アルミナ,窒化ケイ素,サイアロンなどがある.
セラミック最大の欠点は,強度が低く信頼性が小さいことである.大きな力,特に衝撃がかかる場合には容易に破損するため,軽切削などに用途が限られる.そこで,基材に強度の高い工具鋼や超硬合金を用い,その表面をセラミック化することで,耐摩耗性が高くかつ欠けにくい工具にするセラミック・コーティングが行われている.
\section{残留応力測定法について}

残留応力の測定法を以下に示す.

\subsection{X線回折}
金属材料に応力を加えると結晶格子がひずみ,本来の格子間距離より伸縮する.この時X線回折がBragg条件が式(\ref{eq:braggの式})で生じるため,これから格子間距離を求め,残留応力を算出することができる.X線応力測定は非破壊,非接触で表面微小領域の測定や応力分布の可視化,3軸応力解析による主応力の評価などが可能である.
一方,試料表面層の応力しか測定できないことが欠点としてあげられる.

\subsection{ひずみゲージ法}
まず,測定したい個所数方向にひずみゲージを貼り,その部分のひずみを測定.その後,その部分を含んだ小片を切り出し,再度ひずみ測定を行うことで前者との差を求め,応力を測定する方法である.ひずみゲージ法の利点はひずみを直接測定できることや,測定装置が比較的簡単であることなどがあげられる.一方欠点は破壊検査であることであり,小片を切り出す際の塑性変形がひずみゲージに影響を及ぼさないよう考慮する必要がある.
\subsection{穿孔法}
残留応力を有する物体に穴をあけると,その位置で応力が解放される.穴表面でのせん断応力や垂直応力の除去は,すぐ近くの周囲領域の応力を変化させ,それに対応して試験対象物の表面の局所ひずみを変化させる.このひずみをひずみゲージで測定することで残留応力を求めることができる.穿孔法の大きな利点は材料内部の応力分布を計測できることである.欠点は破壊検査であることで,ドリル加工により穴をあけると加工硬化が生じるため,その影響を考慮する必要がある.

\begin{thebibliography}{99}
    \bibitem{ミスミ} ミスミ株式会社,"粒界腐食 | 技術情報 | MISUMI-VONA【ミスミ】"(2024/12/24閲覧)
    \url{https://jp.misumi-ec.com/tech-info/categories/surface_treatment_technology/st01/c1890.html}
    
\end{thebibliography}


\end{document}