\section{考察}
高分子材料の力学特性と微細構造の関係について考察する.急冷した試験片は一度降伏したあと,非常によく伸び,破断応力が大きくなる特徴が見られた.これは組織の大部分が不規則な分子鎖が絡まった非晶質であったことにより,応力に対して組織が柔軟に変形しやすかったことが原因だと考えられる.メカニズムとしては次のようになる.まず,引張応力を加えて塑性変形が始まると,応力が集中した場所でくびれが生じる.その領域では分子鎖の方向が揃っていき,周辺の組織よりも強度が向上する.すると分子鎖間ですべりが生じることでくびれが試験片全体に伝播していき,一部の領域に応力が集中することなく伸びるような挙動が現れる.この一連の変化が急冷した試験片の特性を生じさせたメカニズムだと考えられる.降伏応力が低くなった原因も分子鎖間の分子間力のみで結合しているため変形しやすかったためだと考えられる.一方ヤング率が高くなったのは,急冷による残留応力や分子鎖の整列による応力集中の軽減が影響したと思われる.

65$^\circ$Cで保持した試験片では,降伏応力は最も高くなった.これは微細な球晶をもつことで,隣接する組織に伝播する応力が減少し,塑性変形に必要な応力が増大したとこや,結晶粒界によってき裂や転位の進展が抑制された結果であると考えられる.

120$^\circ$Cで保持した試験片では,65$^\circ$Cで保持したときよりもヤング率,降伏応力ともに低い値を示し,破断するまでのひずみが非常に小さかった.これは,粗大化した球晶を持っていたことで,隣接する組織に伝播する応力が増加し,塑性変形に必要な応力が減少したとこや,結晶粒界の減少によってき裂や転位の進展を抑制する機能が低下した結果であると考えられる.

本実験で作成した試験片の強度が文献値よりも低かったのは,試験片の製作過程で不純物や気泡が混入したことや,引張試験を速く行ったことで材料組織の変形やくびれの伝播が十分に行われなかったことなどが原因だと考えられる.

以上の考察から,材料の強度と延性を両立させて向上させるには,65$^\circ$Cで保持した試験片で現れたような微細な結晶粒を持つ組織を作ることが重要であるといえる.そのような組織を得るためには,図\ref{fig:模式図}において,核形成速度が高く,球晶成長速度が低くなる成形温度に設定することが有効であると考えられる.